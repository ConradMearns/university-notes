\documentclass{article}

\title{Introduction to Leadership in Residence}
\author{Conrad A. Mearns}

\begin{document}

\maketitle

\section{Guiding Statement}
A Resident Assistant's primary role is to build communities by developing relationships with and between residents and foster academic success.

\section{The Seven Vectors of Growth}
Based on theory from Arthur Chickering.
\begin{enumerate}
  \item Developing Competence
    \begin{itemize}
      \item Intellectual - skills and knowledge
      \item Physical - hygene, spatial awareness
      \item Interpersonal - control over emotions, self-control, self expression and self regulation vs repression
    \end{itemize}
  \item Managing Emotions
    \begin{itemize}
      \item Key emphasis on Self Regulation vs Repression of emotions like anger and sadnesses.
    \end{itemize}
  \item Moving Through Autonomy Toward Interdependance
    \begin{itemize}
      \item $Self-sufficiency and responsability \to self chosen goals$
      \item Less care over external opinions
      \item Focus on emotional and instrumental independance
    \end{itemize}
  \item Establishing Identity
    \begin{itemize}
      \item Comfort with body, appearance, gender and sexual orientation
      \item Sense of self socially, historically, and / or culturally in response to feedback
      \item Self acceptance / esteem
      \item Personal stability
    \end{itemize}
  \item Developing Purpose
    \begin{itemize}
      \item Vocational Purpose
      \item Personal Interests
      \item Interpersonal / family commitments
    \end{itemize}
  \item Developing Integrity
    \begin{itemize}
      \item Humanizing Values - showing action behind ethical / moral values
      \item "Liberalization of the super-ego"
      \item Task of higher education institutes like University
      \item Developing personal values that are congruent with socially accepted behaviors
    \end{itemize}
\end{enumerate}

\section{Emotional Intelligence}
Emotioal Intelligence (EI) is ability for individuals to recognize their own, and other people's emotions and to adapt emotion in order to achieve goals. The concept, while heavily criticized, is still considered useful.
\subsection{Five Components of Emotional Intelligence at Work}
\begin{enumerate}
  \item Self-Awareness: The ability to recognize emotions and their effects
  \item Self-Regulation: The ability to control impulses and moods
  \item Motivation: A passion for work that goes beyond money and status
  \item Empathy: The ability to recognize the emotions of others
  \item Social Skill: Proficiency in managine relationships and building networks
\end{enumerate}

\section{Schlossberg's Transition Theory}
Schlossberg defined a transition as any event or non-event that results in a change within our lives.
\subsection{Types of Transitions}
\begin{itemize}
  \item Anticipated transitions: Predictable events like a graduation
  \item Unanticipated transitions: Unpredictable events like divorces and death
  \item Non-events: Transitions that are expected, but cause a transition from something not happening - like failure to get into medical school
\end{itemize}

\subsection{Non-Events}
\begin{itemize}
  \item Personal: related to individual aspirations
  \item Ripple: felt empathetically do to someone else's non-event
  \item Resultant: caused by and event
  \item Delayed: Anticipating and event that might happen
\end{itemize}

\section{Influencing Transitions with the Four S's}
\begin{itemize}
  \item Situation
    \begin{itemize}
      \item Trigger: What caused the transition?
      \item Timing: In terms of the individual's social clock, was transition on time or off time?
      \item Control: What aspects of the transition does the individual control?
      \item Role change: If applicable, is the role change a gain or a loss?
      \item Duration: Is the transition permanent, temporary, or uncertain?
      \item History: Has this or something similar happened before?
      \item Concurrent stress: Are there other sources of stress?
      \item Assessment: Who or what is seen as responsible for the transition and how does that impact behavior?
    \end{itemize}
  \item Self
    \begin{itemize}
      \item Personal and demographic characteristics that effect an individuals views
      \item Ego development, values and commitments
    \end{itemize}
  \item Social Support
    \begin{itemize}
      \item Intimate Relationships
      \item Family
      \item Networks of friends
      \item Institutes and communities
    \end{itemize}
  \item Strategies (for changing / coping)
    \begin{itemize}
      \item that modify the situation
      \item that control the meaning of the problem
      \item that aid in the stress of the aftermath
    \end{itemize}
\end{itemize}

\section{Astin's Theory of Student Involvement}
Student Involvement is defined as the amount of physical and pychological energy that the student devotes to the academic experience. Regardless of where involvement is focused, some students will spend more energy than others. Involvement can be measured quantitatively (amount of time devoted) and qualitatively (seriousness with what an object or activity was met with).

\section{Boyer's Community}
\begin{enumerate}
  \item Purposeful
  \item Open
  \item Just
  \item Disciplined
  \item Caring
  \item Celebrative
\end{enumerate}

\section{Intentional Interaction Model}
There is a need for guidance in new students, interacting with them improves student retention and success. Intentional interactions are interactions with purpose to help students make new connections, get to know students, form communities, and improve personal wellness.



\end{document}
