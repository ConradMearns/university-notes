\documentclass{article}

\title{Analytical Geometry and Calculus II}
\author{Conrad A. Mearns}

\begin{document}

\maketitle

\noindent
\Large
January 12, 2017\\\\
Review of Limits\\
\normalsize

\noindent
Theorems for Limits:\\
Let $a$ and $c$ be any number, if $F = \lim_{x \to a}f(x)$ and $G = \lim_{x \to a}g(x)$ then

\begin{enumerate}
  \item $\lim_{x \to a}(f(x) + g(x)) = F + G$
  \item $\lim_{x \to a}(f(x) - g(x)) = F - G$
  \item $\lim_{x \to a}(c * f(x)) = c * F$
  \item $\lim_{x \to a}(\frac{f(x)}{g(x)}) = \frac{\lim_{x \to a}f(x)}{\lim_{x \to a}g(x)}$ except when $G = 0$
  \item $\lim_{x \to a}(f(x))^c = F^c$
\end{enumerate}

\noindent
\Large
Limits Involving Infinity while $c \neq 0$\\
\normalsize
\noindent
These are templates, where $x$ is either taken to $\infty$ or 0.

\begin{enumerate}
  \item $c * (\pm \infty) = \pm \infty$ Example: $\lim_{x \to \infty}5x = \infty$
  \item $\frac{c}{\pm \infty} = 0$ Example: $\lim_{x \to \infty}\frac{5}{x} = 0$
  \item $\frac{c}{0} = \pm \infty$ Example: $\lim_{x \to 0}\frac{5}{0} = \infty$
  \item $\frac{\pm \infty}{c} = \pm \infty$
\end{enumerate}

\noindent
\Large
Limits that can't be dealt with
\normalsize
\noindent
\begin{itemize}
  \item $\frac{0}{0}$
  \item $\frac{\infty}{\infty}$
  \item $\infty - \infty$
  \item $0 * \infty$
\end{itemize}

\noindent
\Large
Limits of Rational Functions\\
\normalsize
\noindent
Let $d$ represent the degree of $f(x)$ and $e$ represent the degree of $g(x)$.\\
Theorem: Given a rational function $\frac{f(x)}{g(x)}$, the following is be true.
\begin{enumerate}
  \item If $d > e$ then $\lim_{x \to \infty}\frac{f(x)}{g(x)} = \pm \infty$
  \item If $d < e$ then $\lim_{x \to \infty}\frac{f(x)}{g(x)} = 0$
  \item If $d = e$ then $\lim_{x \to \infty}\frac{f(x)}{g(x)} = \frac{a}{b}$ where $a$ is the leading term of $f(x)$ and $b$ is the leading term of $g(x)$
\end{enumerate}

\noindent
Example: To evalute $lim_{x \to \infty}\frac{10x^3 - 3x^2 + 8}{\sqrt{25x^6 + x^4 + 2}}$ first reduce all terms by $x$ of the leading coefficient's degree.\\
$\frac{10 - 3x^{-1} + 8x^{-3}}{\sqrt{25 + x^{-2} + 2x^{-6}}}$ Then take the limit of each term.\\
$\frac{10 - 0 + 0}{\sqrt{25 + 0 + 0}}$ Simplifiy.\\
$\frac{10}{\sqrt{25}} = \frac{10}{5} = 2$\\

\noindent
\Large Inverse Trig Functions\\
\normalsize
\noindent
Definition: $y = sin^{-1}x$ is the value of $y$ such that $x = siny$.\\
\indent
Domain: $-1 \leq x \leq 1$\\
\indent
Range: $\frac{-\pi}{2} \leq y \leq \frac{\pi}{2}$\\

\noindent
Definition: $y = cos^{-1}x$ is the value of $y$ such that $x = cosy$.\\
\indent
Domain: $-1 \leq x \leq 1$\\
\indent
Range: $0 \leq y \leq \pi$\\

\noindent
\Large Inverse Trig Identities
\normalsize
\noindent
\begin{itemize}
  \item $sin(sin^{-1}x) = x$
  \item $cos(cos^{-1}x) = x$
  \item $sin^{-1}(sinx) = x$ only if $x$ is in range of $sin^{-1}$
  \item $cos^{-1}(cosx) = x$ only if $x$ is in range of $cos^{-1}$
\end{itemize}

\noindent
Example: $sin^{-1}(sin\pi) = sin^{-1}(0) = 0 \neq \pi$\\

\noindent
Example: $cos(sin^{-1}x)$\\
Let $y = sin^{-1}x$ so that $x = siny$ and $cos(sin^{-1}x) = cosy$\\
Recall that $sin = \frac{opposite}{hypotenuse}$\\
Let $hypotenuse = 1$ and $opposite = b$ where $b$ has yet to be determined.\\
Recall that $cosy = \frac{adjacent}{hypotenuse}$\\
$\frac{adjacent}{hypotenuse} = \frac{b}{1}$, and therefor $cosy = b$\\
Use the Pythagorean Theorem to solve: $x^2 + b^2 = 1^2$\\
$b^2 = 1 - x^2$\\
$b = \sqrt{1 - x^2}$\\
Therefor $cos(sin^{-1}x) = \sqrt{1 - x^2}$


\end{document}
