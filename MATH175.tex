\documentclass{article}

\title{Analytical Geometry and Calculus II}
\author{Conrad A. Mearns}

\begin{document}

\maketitle

\noindent
\Large
Review of Theorems for Limits\\
\normalsize

\noindent
Let $a$ and $c$ be any number, if $F = \lim_{x \to a}f(x)$ and $G = \lim_{x \to a}g(x)$ then

\begin{enumerate}
  \item $\lim_{x \to a}(f(x) + g(x)) = F + G$
  \item $\lim_{x \to a}(f(x) - g(x)) = F - G$
  \item $\lim_{x \to a}(c * f(x)) = c * F$
  \item $\lim_{x \to a}(\frac{f(x)}{g(x)}) = \frac{\lim_{x \to a}f(x)}{\lim_{x \to a}g(x)}$ except when $G = 0$
  \item $\lim_{x \to a}(f(x))^c = F^c$
\end{enumerate}

\noindent
\Large
Limits Involving Infinity while $c \neq 0$\\
\normalsize
\noindent
These are templates, where $x$ is either taken to $\infty$ or 0.

\begin{enumerate}
  \item $c * (\pm \infty) = \pm \infty$ Example: $\lim_{x \to \infty}5x = \infty$
  \item $\frac{c}{\pm \infty} = 0$ Example: $\lim_{x \to \infty}\frac{5}{x} = 0$
  \item $\frac{c}{0} = \pm \infty$ Example: $\lim_{x \to 0}\frac{5}{0} = \infty$
  \item $\frac{\pm \infty}{c} = \pm \infty$
\end{enumerate}

% \noindent
% \Large
% Impossible Limits
% \normalsize
% \noindent
% \begin{itemize}
%   \item $\frac{0}{0}$
%   \item $\frac{\infty}{\infty}$
%   \item $\infty - \infty$
%   \item $0 * \infty$
% \end{itemize}

\noindent
\Large
Limits of Rational Functions\\
\normalsize
\noindent
Theorem: Given a rational function $\frac{f(x)}{g(x)}$, the following is be true.
Let $d$ represent the degree of $f(x)$ and $e$ represent the degree of $g(x)$.\\
\begin{enumerate}
  \item If $d > e$ then $\lim_{x \to \infty}\frac{f(x)}{g(x)} = \pm \infty$
  \item If $d < e$ then $\lim_{x \to \infty}\frac{f(x)}{g(x)} = 0$
  \item If $d = e$ then $\lim_{x \to \infty}\frac{f(x)}{g(x)} = \frac{a}{b}$ where $a$ is the leading term of $f(x)$ and $b$ is the leading term of $g(x)$
\end{enumerate}

\noindent
Example: To evalute $lim_{x \to \infty}\frac{10x^3 - 3x^2 + 8}{\sqrt{25x^6 + x^4 + 2}}$
\begin{enumerate}
  \item {Reduce all terms by $x$ of the leading coefficient's degree.}
  \item {Then take the limit of each term. $\frac{10 - 3x^{-1} + 8x^{-3}}{\sqrt{25 + x^{-2} + 2x^{-6}}}$}
  \item {Simplifiy. $\frac{10 - 0 + 0}{\sqrt{25 + 0 + 0}}$}
  \item {$\frac{10}{\sqrt{25}} = \frac{10}{5} = 2$}
\end{enumerate}

\noindent
\Large Inverse Trig Functions\\
\normalsize
\noindent
Definition: $y = sin^{-1}x$ is the value of $y$ such that $x = siny$.\\
\indent
Domain: $-1 \leq x \leq 1$\\
\indent
Range: $\frac{-\pi}{2} \leq y \leq \frac{\pi}{2}$\\

\noindent
Definition: $y = cos^{-1}x$ is the value of $y$ such that $x = cosy$.\\
\indent
Domain: $-1 \leq x \leq 1$\\
\indent
Range: $0 \leq y \leq \pi$\\

\noindent
Other Trig Functions
\begin{itemize}
  \item {
  $y = tan^{-1}x \to x = tany$\\
  Range: $\frac{-\pi}{2} < y < \frac{\pi}{2}$
  }
  \item {
  $y = cot^{-1}x \to x = coty$\\
  Range: $0 < y < \pi$
  }
  \item {
  $y = sec^{-1}x \to x = secy$\\
  Range: $0 \leq y \leq \pi, y \neq\frac{\pi}{2}$
  }
  \item {
  $y = csc^{-1}x \to x = cscy$\\
  Range: $\frac{-\pi}{2} \leq y \leq \frac{\pi}{2}, y \neq 0$
  }
\end{itemize}

\noindent
\Large Inverse Trig Identities
\normalsize
\noindent
\begin{itemize}
  \item $sin(sin^{-1}x) = x$
  \item $cos(cos^{-1}x) = x$
  \item $sin^{-1}(sinx) = x$ only if $x$ is in range of $sin^{-1}$
  \item $cos^{-1}(cosx) = x$ only if $x$ is in range of $cos^{-1}$
\end{itemize}

\noindent
Example: $sin^{-1}(sin\pi) = sin^{-1}(0) = 0 \neq \pi$\\

\noindent
Example: $cos(sin^{-1}x)$
\begin{enumerate}
  \item {Let $y = sin^{-1}x$ so that $x = siny$ and $cos(sin^{-1}x) = cosy$\\}
  \item {Recall that $sin = \frac{opposite}{hypotenuse}$\\}
  \item {Let $hypotenuse = 1$ and $opposite = b$ where $b$ has yet to be determined.\\}
  \item {Recall that $cosy = \frac{adjacent}{hypotenuse}$\\}
  \item {$\frac{adjacent}{hypotenuse} = \frac{b}{1}$, and therefor $cosy = b$\\}
  \item {Use the Pythagorean Theorem to solve: $x^2 + b^2 = 1^2$\\}
  \item {$b^2 = 1 - x^2$\\}
  \item {$b = \sqrt{1 - x^2}$\\}
  \item {Therefor $cos(sin^{-1}x) = \sqrt{1 - x^2}$}
\end{enumerate}

\noindent
\Large Derivatives of Inverse Trig Functions
\normalsize
\noindent

\begin{itemize}
  \item {
  $\frac{d}{dx}(sin^{-1}x) = \frac{1}{\sqrt{1-x^2}}$
  }
  \item {
  $\frac{d}{dx}(cos^{-1}x) = \frac{-1}{\sqrt{1-x^2}}$
  }
  \item {
  $\frac{d}{dx}(tan^{-1}x) = \frac{1}{1+x^2}$
  }
  \item {
  $\frac{d}{dx}(cot^{-1}x) = \frac{-1}{1-x^2}$
  }
  \item {
  $\frac{d}{dx}(sec^{-1}x) = \frac{1}{|x|\sqrt{x^2-1}}$
  }
  \item {
  $\frac{d}{dx}(csc^{-1}x) = \frac{-1}{|x|\sqrt{x^2-1}}$
  }
\end{itemize}

\noindent
\Large Antiderivatives Involving Inverse Trig Functions
\normalsize
\noindent

\begin{itemize}
  \item {
  $\int\frac{dx}{\sqrt{a^2 - x^2}} = sin^{-1}\frac{x}{a}+C$
  }
  \item {
  $\int\frac{dx}{a^2 + x^2} = \frac{1}{a}tan^{-1}\frac{x}{a}+C$
  }
  \item {
  $\int\frac{dx}{x\sqrt{x^2 - a^2}} = \frac{1}{a}sec^{-1}\frac{x}{a}+C$
  }
\end{itemize}

\noindent
\Large L'Hopital's Rule\\
\normalsize
\noindent

L'Hopital's Rule let's us evaluate impossible limits.\\\\

\noindent
Theorem: Suppose $f(x)$ and $g(x)$ are differentiable on an open interval $I$ containing $a$ where $g'(x) \neq 0$ on I when $x \neq a$. If
\begin{enumerate}
  \item $\lim_{x \to a}f(x) = \lim_{x \to a}g(x) = 0$
  \item $\lim_{x \to a}f(x) = \pm\infty$ and $\lim_{x \to a}g(x) = \pm\infty$
\end{enumerate}
then $\lim_{x \to a}\frac{f(x)}{g(x)} = \lim_{x \to a}\frac{f'(x)}{g'(x)}$. This is also true as $x \to \pm\infty, x \to a^+, x \to a^-$.

\noindent
\Large
Basic Approaches to Integration\\
\normalsize
\begin{itemize}
  \item {
  Subtle Substitution\\
  $\int\frac{dx}{x^{-1}+1}$\\
  $=\int\frac{x}{x+1}dx$ and suppose $u = x + 1, x = u - 1, du = dx$\\
  $=\int\frac{u-1}{u}du$\\
  $=\int du - \int \frac{1}{u}du = u - \ln|u| + C$\\
  $= x+1 - ln|x| + C$
  }
  \item {
  Splitting Fractions\\
  $\int\frac{2-3x}{\sqrt{1-x^2}}dx$\\
  $=\int\frac{2}{\sqrt{1-x^2}}dx - \frac{3x}{\sqrt{1-x^2}}dx$\\
  $=\int 2\sin^{-1}x - \frac{3x}{\sqrt{1-x^2}}dx$\\
  etc...
  }
  \item {
  Completing the Square\\
  $\int\frac{dx}{\sqrt{27 - 6x - x^2}}$\\
  $=\int\frac{dx}{\sqrt{-(x^2 +6x -27)}}$\\
  $=\int\frac{dx}{\sqrt{-((x+3)^2 -36)}}$\\
  $=\int\frac{dx}{\sqrt{36 - (x+3)^2}}$\\
  $=\sin^{-1}(\frac{x+3}{6}) + C$\\
  }
  \item {
  Multiplying by 1 (Using Conjugates)\\
  $\int\frac{dx}{1 + \sin x}$ the conjugate of $1 + \sin x$ is $1 - \sin x$\\
  $=\int\frac{dx (1 - \sin x)}{(1 + \sin x)(1 - \sin x)}$\\
  $=\int\frac{1 - \sin x}{1 - \sin^2 x}dx$\\
  $=\int\frac{1 - \sin x}{\cos^2 x}dx$\\
  $=\int\frac{1}{\cos^2 x}dx - \int\frac{\sin x}{\cos^2 x}dx$\\
  $=\int \sec^2 x dx - \int \tan x \sec x dx$\\
  $=\tan x - \sec x + C$\\
  }
\end{itemize}

\noindent
\Large
Integration by Parts\\
\normalsize
\indent
Integration by Parts, or IBP, is used when integrating products of functions. It's not perfect, and can get messy if used incorrectly. The basic form is\\

$\int{udv} = uv - \int{vdu}$\\

Solve by substituting $u$ and $dv$, then using the right hand form.\\

\noindent
Example: $\int{t e^t dt}$
\begin{enumerate}
  \item Let $u = t$ and $dv = e^t dt$ so that $du = dt$ and $v = e^t$
  \item $= t e^t - \int{e^t dt}$
  \item $= t e^t - e^t + C$
\end{enumerate}

\noindent
\Large
Trigometetric Integrals\\
\normalsize
Products of sin and cos
\begin{itemize}
  \item If the power of sin or cos split off 1 factor and use $\sin^2x + \cos^2x = 1$\\
  Example: $\int{\cos^3x dx} \to \int{\cos^2x \cos x} \to \int{(1-\sin^2x)\cos x dx}$
  \item If the power of sin or cos is even, use a half-angle identity.\\
  $\cos^2x = \frac{1 + \cos^2x}{2}$ and $\sin^2x = \frac{1 - \sin^2x}{2}$\\
  Example: $\int{\sin^2x dx} = \int{\frac{1- \cos^2x}{2}dx}$\\
  $= \int{\frac{1}{2}dx} - \int{\frac{1}{2}\cos^2x dx}$\\
  $= \frac{x}{2} - \frac{1}{4}\sin 2x + C$
\end{itemize}

\noindent
Products of of powers of sin and cos
\begin{itemize}
  \item If the power of sinx or cosx is odd, split off a factor and rewrite the resulting power in terms of the opposite, then use U-Substituion.
\end{itemize}

\noindent
Powers of tan, sec, cot and csc
\begin{itemize}
  \item $\int{\sec^2x} = \tan x + C$
  \item $\int{\tan^2x} = \int{\sec^2x + 1} = \tan x + x + C$
  \item $\int{\csc^2x} = -\cot x + C$
  \item $\int{\cot^2x} = \int{\csc^2x + 1} = -\cot x + x + C$
\end{itemize}

\noindent
Products of Powers of tan and sec
\begin{itemize}
  \item If the power of sec is even, split $\sec^2$, rewrite in terms of tanx in terms of sec then use U-Sub on $\tan x$.\\
  Example: $\int{\sec^2x\tan^{1/2}x dx}$\\
  Let $u = \tan{x}$ and $du = \sec^2{x}$\\
  $= \int{u^{1/2}du}$\\
  $= \frac{2}{3}u^{3/2} + C$\\
  \item If the power of tan is odd, split off secxtanx, rewrite remaining even power of tanx in terms of secx then use U-Sub on $\sec{x}$.\\
  Example: $10\int{\tan^9{x}\sec^2{x}dx} = 10\int{\tan^8{x}\sec{x}\sec{x}\tan{x}dx}$\\
  $= 10\int{(\sec^2{x} - 1)^4 \sec{x}\sec{x}\tan{x} dx}$\\
  $u = \sec{x}$ and $du = \sec{x}\tan{x}dx$\\
  $= 10\int{(u^2 - 1)^4 u du}$\\
\end{itemize}

\noindent
\Large
Integrating with Trig Substitutions
\normalsize
\noindent
For forms of $a^2 - x^2$, $a^2 + x^2$ and $x^2 - a^2$ - because powers do not distribute ovr sums / differences.\\
$a^2 - x^2$ by substituting $x = a\sin{\theta}$
\begin{itemize}
  \item
\end{itemize}

\noindent
\Large
Partial Functions\\
\normalsize
\indent
Integrating rational functions is typically complicated. Often times they can be rewritten as if the function was created as a some of smaller rational functions. Suppose $\frac{x+2}{x^3 -3x^2 +2x}$. First rewrite the denominator so that each $x$ term can be solved.\\
$=\frac{x+2}{x(x-2)(x-1)}$\\
We assume that $=\frac{x+2}{x(x-2)(x-1)} = \frac{A}{x} + \frac{B}{x-2} + \frac{C}{x-1}$ in order to split the fraction up\\
Multiply every term by the denominators. $x+2 = A(x-2)(x-1) + Bx(x-1) + Cx(x-2)$\\
Solve by substituting $x$ with numbers to get 0 terms.\\
$x=2 \to 4 = A(0)(1) + 2B(1) + 2C(0) \to 4 = 2B \to 2 = B$\\
$x=1 \to 3 = A(-1)(0) + B(0) + C(-1) \to 3 = -C \to -3 = C$\\
$x=0 \to 2 = A(-2)(-1) + 0B(-1) + 0C(-2) \to 2 = 2A \to 1 = A$\\
Therefor\\
$=\frac{x+2}{x(x-2)(x-1)} = \frac{1}{x} + \frac{2}{x-2} + \frac{-3}{x-1}$\\

\noindent
\Large
Partial Fraction Decomposition (PFD) - Irreducible Quadratic Factors\\
\normalsize
\indent
The denomonator $d$ is a root of the quadratic function $f(x)$ in and only if $x-d$ is a factor of $f(x)$. Therefor $f(x)$ is irreducible when $b^2 - 4ac < 0$\\
Example: $\int{\frac{x^2 + x + 2}{(x + 1)(x^2 + 1)}}dx$\\
$x + 1$ and $x^2 + 1$ are irreducible so use PFD\\
$\frac{x^2 + x + 2}{(x + 1)(x^2 + 1)} = \frac{A}{x+1} + \frac{Bx + C}{x^2 + 1}$\\
$x^2 + x + 2 = A(x^2 + 1) + (Bx + C)(x + 1)$\\
$x = -1 \to 2 = 2A \to 1 = A$\\
$x^2 + x + 2 = x^2 + 1 + (Bx + C)(x + 1)$\\
$x = 0 \to 2 = 1 + C \to 1 = C$\\
$x^2 + x + 2 = x^2 + 1 + (Bx + 1)(x + 1)$\\
$x + 2 = 1 + (Bx + 1)(x + 1)$\\
$x + 2 = Bx^2 + Bx + x + 2$\\
$0 = Bx^2 + Bx$\\
$0 = B$\\
Therefor $\int{\frac{x^2 + x + 2}{(x + 1)(x^2 + 1)}} = \int{\frac{1}{x + 1} + \frac{1}{x^2 + 1}}dx$\\

\noindent
\Large
Numerical Integration\\
\normalsize
\indent
There are three methods generally used to approximate definite integerals. All three rely on splitting the interval into smaller sub-regions whose area is more easliy found.

For all three rules, use an arbitray $n$ value. Over the interval $[a, b]$, define $\delta x = \frac{b - a}{n}$\\
Midpoint Rule\\
$Area \approx \delta x (\sum_{k=1}^{n} f(\frac{x_{k-1} + x_k}{2}))$\\
Trapezoid Rule\\
$Area \approx \delta x [\frac{f(x)}{2} + (\sum_{k=1}^{n-1} f(x_k)) + \frac{f(x_n)}{2}]$\\
Simpson's Rule\\
$Area \approx \frac{\delta x}{3}(f(x_0) + 4f(x_1) + 2f(x_2) + ... + 4f(x_{n-1}) + f(x_n))$\\

Denote usage of these rules - midpoint, trapezoid, and simpson's - with $n$ subintervals as following.\\
$M(n) = \delta x(f(\frac{x_0 + x_1}{2}) + ... + f(\frac{x_{n-1} + x_n}{2}))$\\
$T(n) = \delta x(\frac{1}{2} f(x_0) + f(x_1) + ... + f(x_{n-1}) + \frac{1}{2} f(x_n))$\\
$S(n) = \frac{\delta x}{3}(f(x_0) + 4f(x_1) + 2f(x_2) + ... + 4f(x_{n-1}) + f(x_n))$\\

\noindent
\Large
Numerical Integration Shortcuts\\
\normalsize
\indent
$T(2n) = \frac{T(n) + M(n)}{2}$\\
$S(2n) = \frac{4T(2n) - T(n)}{3}$\\

\noindent
\Large
Errors in Approximating with Numerical Integration\\
\normalsize
\indent
Let $E_M$ be the error using the midpoint rule over the interval $[a, b]$\\
$E_M = \frac{k(b-a)}{24}(\delta x)^2$\\
Let $E_T$ be the error using the trapezoid rule over the interval $[a, b]$\\
$E_M = \frac{k(b-a)}{12}(\delta x)^2$\\
Let $E_S$ be the error using Simpson's rule over the interval $[a, b]$\\
$E_M = \frac{K(b-a)}{180}(\delta x)^4$\\

\noindent
\Large
Sequences\\
\normalsize
\indent
A sequence is an infinite list of numbers like $1, 2, 3, 4 ...$. We define sequences by patterns, forumlas, or recurrence relations.\\
For example, $1, 2, 4, 8, 16, 32 ...$ has an explicit forumula of $A_n = 2^n$ for $n \geq 0$\\
The arbitrary sequence $(a_1, a_2, ...)$ is denoted as $(a_n)_{n=1}$ or just $a_n$. Thus\\
$(1, 2, 3, 4, ...) = (n)$\\
$(1, 2, 4, 8, 16, ...) = (2^{n-1})$\\
Recurrence relations can also define a sequence.\\
$a_1 = 1$ and $a_n = 2a_{n-1}$ for $n \geq 2$ gives us $1, 2, 4, 8, ...$\\
Theorem: Let $f$ be a function with $f(n) = a_n$ for all positive integer values of $n$. If $\lim_{n \to \infty} f(n) = L$ then $\lim_{n \to \infty} a_n = L$\\
If $\lim_{n \to \infty} a_n = L$ then $a_n$ converges to $L$ and we write $(a_n) \to L$\\
We can then define the following to be true
\begin{itemize}
  \item If $a_n \to A$ and $b_n \to B$ then $a_n \pm b_n \to A \pm B$
  \item $ca_n \to cA$
  \item $a_n * b_n \to AB$
  \item $\frac{a_n}{b_n} \to \frac{A}{B}$
\end{itemize}

We also define the following terms as follows
\begin{itemize}
  \item $a_n$ is increasing if $a_{n+1} \geq a_n$ for all $n$
  \item $a_n$ is decreasing if $a_{n+1} \leq a_n$ for all $n$
  \item $a_n$ is monotonic if $a_{n+1}$ is either increasing or decreasing
  \item $a_n$ is bounded if there is a number $M$ such that $|a_n| \leq M$ for all $n$
\end{itemize}

Theorem: Bounded monotonic sequences converge\\
Let $M$ be the smallest bound. If $(a_n)$ goes in one direction and is bounded by $M$ then $(a_n)$ will approach but never reach $M$ as $n \to \infty$

\noindent
\Large
Geometric Sequences\\
\normalsize
\indent
A Geometric Sequence is any sequence of the form $(a * r^n)$ where $a$ and $r$ are numbers. If $(a * r^n)$ converges, $a$ is not needed to evaluate the bound (given by limit laws).
\begin{itemize}
  \item If $(r^n) \to L$ then $(a * r^n) \to aL$
  \item If $(a* r^n) \to M$ then $(\frac{1}{a} * r^n) \to \frac{M}{a}$
\end{itemize}

Definition: We say some property $P(n)$ holds eventually if there is an integer $k$ such that $P(n)$ holds true for all $n \geq k$\\
Therefor eventually $a_n \geq k$ if $(a_n) = (n)$

\noindent
\Large
Squeeze Theorem\\
\normalsize
\indent
An (infinite) series is an infinite sum of numbers\\
$a_1 + a_2 + a_3 + ... = \sum_{k=1}^{\infty} a_k$\\
$\sum_{k=1}^{\infty} a_k = \lim_{n \to \infty}\sum_{k=1}^{n} a_k = (a_1 + a_2 + a_3 + ...)$\\
Denote $\sum_{k=1}^{n} a_k$ by $S_n$ which is called the nth partial.
\begin{itemize}
  \item $S_1 = a_1$
  \item $S_2 = a_1 + a_2$
  \item $...$
  \item $S_n = a_1 + a_2 + ... + a_n$
\end{itemize}

We define the limit of $\sum_{k=1}^{\infty}a_k$ to be $\lim_{n \to \infty}S_n = \sum_{k=1}^{n}a_k$\\
If $\lim_{n \to \infty}(S_n) = L$ we write $\sum_{k=1}^{\infty}a_k = L$\\
If $L$ exists then $\sum_{k=1}^{\infty} a_k$ converges. If not, $\sum_{k=1}^{\infty}a_k$ diverges.

If $(a_n) = (c)$ then $S_n = nc$ so $S_n \to \infty$ which means non-monotonic sequences cannot produce converging series. To have a converging series given a sequence, $(a_n) \to 0$ must be true and it must converge quickly.\\
Example: $\sum_{k=1}^{\infty}\frac{1}{n}$ diverges only because it is not quick enough.


\end{document}
