\documentclass{article}

\title{Analytical Geometry and Calculus II}
\author{Conrad A. Mearns}

\begin{document}

\maketitle

\noindent
\Large
Review of Theorems for Limits\\
\normalsize

\noindent
Let $a$ and $c$ be any number, if $F = \lim_{x \to a}f(x)$ and $G = \lim_{x \to a}g(x)$ then

\begin{enumerate}
  \item $\lim_{x \to a}(f(x) + g(x)) = F + G$
  \item $\lim_{x \to a}(f(x) - g(x)) = F - G$
  \item $\lim_{x \to a}(c * f(x)) = c * F$
  \item $\lim_{x \to a}(\frac{f(x)}{g(x)}) = \frac{\lim_{x \to a}f(x)}{\lim_{x \to a}g(x)}$ except when $G = 0$
  \item $\lim_{x \to a}(f(x))^c = F^c$
\end{enumerate}

\noindent
\Large
Limits Involving Infinity while $c \neq 0$\\
\normalsize
\noindent
These are templates, where $x$ is either taken to $\infty$ or 0.

\begin{enumerate}
  \item $c * (\pm \infty) = \pm \infty$ Example: $\lim_{x \to \infty}5x = \infty$
  \item $\frac{c}{\pm \infty} = 0$ Example: $\lim_{x \to \infty}\frac{5}{x} = 0$
  \item $\frac{c}{0} = \pm \infty$ Example: $\lim_{x \to 0}\frac{5}{0} = \infty$
  \item $\frac{\pm \infty}{c} = \pm \infty$
\end{enumerate}

\noindent
\Large
Impossible Limits
\normalsize
\noindent
\begin{itemize}
  \item $\frac{0}{0}$
  \item $\frac{\infty}{\infty}$
  \item $\infty - \infty$
  \item $0 * \infty$
\end{itemize}

\noindent
\Large
Limits of Rational Functions\\
\normalsize
\noindent
Theorem: Given a rational function $\frac{f(x)}{g(x)}$, the following is be true.
Let $d$ represent the degree of $f(x)$ and $e$ represent the degree of $g(x)$.\\
\begin{enumerate}
  \item If $d > e$ then $\lim_{x \to \infty}\frac{f(x)}{g(x)} = \pm \infty$
  \item If $d < e$ then $\lim_{x \to \infty}\frac{f(x)}{g(x)} = 0$
  \item If $d = e$ then $\lim_{x \to \infty}\frac{f(x)}{g(x)} = \frac{a}{b}$ where $a$ is the leading term of $f(x)$ and $b$ is the leading term of $g(x)$
\end{enumerate}

\noindent
Example: To evalute $lim_{x \to \infty}\frac{10x^3 - 3x^2 + 8}{\sqrt{25x^6 + x^4 + 2}}$
\begin{enumerate}
  \item {Reduce all terms by $x$ of the leading coefficient's degree.}
  \item {Then take the limit of each term. $\frac{10 - 3x^{-1} + 8x^{-3}}{\sqrt{25 + x^{-2} + 2x^{-6}}}$}
  \item {Simplifiy. $\frac{10 - 0 + 0}{\sqrt{25 + 0 + 0}}$}
  \item {$\frac{10}{\sqrt{25}} = \frac{10}{5} = 2$}
\end{enumerate}

\noindent
\Large Inverse Trig Functions\\
\normalsize
\noindent
Definition: $y = sin^{-1}x$ is the value of $y$ such that $x = siny$.\\
\indent
Domain: $-1 \leq x \leq 1$\\
\indent
Range: $\frac{-\pi}{2} \leq y \leq \frac{\pi}{2}$\\

\noindent
Definition: $y = cos^{-1}x$ is the value of $y$ such that $x = cosy$.\\
\indent
Domain: $-1 \leq x \leq 1$\\
\indent
Range: $0 \leq y \leq \pi$\\

\noindent
Other Trig Functions
\begin{itemize}
  \item {
  $y = tan^{-1}x \to x = tany$\\
  Range: $\frac{-\pi}{2} < y < \frac{\pi}{2}$
  }
  \item {
  $y = cot^{-1}x \to x = coty$\\
  Range: $0 < y < \pi$
  }
  \item {
  $y = sec^{-1}x \to x = secy$\\
  Range: $0 \leq y \leq \pi, y \neq\frac{\pi}{2}$
  }
  \item {
  $y = csc^{-1}x \to x = cscy$\\
  Range: $\frac{-\pi}{2} \leq y \leq \frac{\pi}{2}, y \neq 0$
  }
\end{itemize}

\noindent
\Large Inverse Trig Identities
\normalsize
\noindent
\begin{itemize}
  \item $sin(sin^{-1}x) = x$
  \item $cos(cos^{-1}x) = x$
  \item $sin^{-1}(sinx) = x$ only if $x$ is in range of $sin^{-1}$
  \item $cos^{-1}(cosx) = x$ only if $x$ is in range of $cos^{-1}$
\end{itemize}

\noindent
Example: $sin^{-1}(sin\pi) = sin^{-1}(0) = 0 \neq \pi$\\

\noindent
Example: $cos(sin^{-1}x)$
\begin{enumerate}
  \item {Let $y = sin^{-1}x$ so that $x = siny$ and $cos(sin^{-1}x) = cosy$\\}
  \item {Recall that $sin = \frac{opposite}{hypotenuse}$\\}
  \item {Let $hypotenuse = 1$ and $opposite = b$ where $b$ has yet to be determined.\\}
  \item {Recall that $cosy = \frac{adjacent}{hypotenuse}$\\}
  \item {$\frac{adjacent}{hypotenuse} = \frac{b}{1}$, and therefor $cosy = b$\\}
  \item {Use the Pythagorean Theorem to solve: $x^2 + b^2 = 1^2$\\}
  \item {$b^2 = 1 - x^2$\\}
  \item {$b = \sqrt{1 - x^2}$\\}
  \item {Therefor $cos(sin^{-1}x) = \sqrt{1 - x^2}$}
\end{enumerate}

\noindent
\Large Derivatives of Inverse Trig Functions
\normalsize
\noindent

\begin{itemize}
  \item {
  $\frac{d}{dx}(sin^{-1}x) = \frac{1}{\sqrt{1-x^2}}$
  }
  \item {
  $\frac{d}{dx}(cos^{-1}x) = \frac{-1}{\sqrt{1-x^2}}$
  }
  \item {
  $\frac{d}{dx}(tan^{-1}x) = \frac{1}{1+x^2}$
  }
  \item {
  $\frac{d}{dx}(cot^{-1}x) = \frac{-1}{1-x^2}$
  }
  \item {
  $\frac{d}{dx}(sec^{-1}x) = \frac{1}{|x|\sqrt{x^2-1}}$
  }
  \item {
  $\frac{d}{dx}(csc^{-1}x) = \frac{-1}{|x|\sqrt{x^2-1}}$
  }
\end{itemize}

\noindent
\Large Antiderivatives Involving Inverse Trig Functions
\normalsize
\noindent

\begin{itemize}
  \item {
  $\int\frac{dx}{\sqrt{a^2 - x^2}} = sin^{-1}\frac{x}{a}+C$
  }
  \item {
  $\int\frac{dx}{\sqrt{a^2 + x^2}} = \frac{1}{a}tan^{-1}\frac{x}{a}+C$
  }
  \item {
  $\int\frac{dx}{x\sqrt{x^2 - a^2}} = \frac{1}{a}sec^{-1}\frac{x}{a}+C$
  }
\end{itemize}

\noindent
\Large L'Hopital's Rule\\
\normalsize
\noindent

L'Hopital's Rule let's us evaluate impossible limits.\\\\

\noindent
Theorem: Suppose $f(x)$ and $g(x)$ are differentiable on an open interval $I$ containing $a$ where $g'(x) \neq 0$ on I when $x \neq a$. If
\begin{enumerate}
  \item $\lim_{x \to a}f(x) = \lim_{x \to a}g(x) = 0$
  \item $\lim_{x \to a}f(x) = \pm\infty$ and $\lim_{x \to a}g(x) = \pm\infty$
\end{enumerate}
then $\lim_{x \to a}\frac{f(x)}{g(x)} = \lim_{x \to a}\frac{f'(x)}{g'(x)}$. This is also true as $x \to \pm\infty, x \to a^+, x \to a^-$.

\end{document}
