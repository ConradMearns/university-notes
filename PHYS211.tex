\documentclass{article}

\title{Engineering Physics I}
\author{Conrad A. Mearns}

\begin{document}

\maketitle

\noindent
\Large
Significant Figures\\
\normalsize
\noindent
When multiplying or dividing, the result is as precise as the least precise input to the number of digits.\\
Example: $54.3 * 6.8991 = 374.62113$, truncated to $374$ or rounded to $375$.\\\\

\noindent
When adding or subtracting, the result is as precise as the least precise input to the nuber of digits post decimal place.\\
Example: $10.65 + 3.0 = 13.65$, truncated to $13.6$ or rounded to $13.7$.\\\\

\noindent
As a general rule (from the professor), round up down to the nearset even number, as always rounding up will accumulate more error.

\noindent
\Large
Variables of Movement and Position\\
\normalsize

\begin{enumerate}
  \item Position: Location in space with respect to another object or coordinate system.\\
    $x, y, z$
  \item Displacement: Difference in position at two different times.\\
    $\Delta x, \Delta y, \Delta z, \Delta x = x_2 - X_1$
  \item Average Velocity: Displacement divided by time.\\\
    $v_{avg}, v_{avg} = \frac{\Delta x}{\Delta t}$
  \item Speed: Total distance divided by time.\\
    $s, s \equiv \frac{d}{t}$
  \item Instantanious Velocity: Velocity measured at a single time.\\
    $v, v = \lim_{t \to a}\frac{\Delta x}{\Delta t} = \lim_{t \to a}\frac{x_2 - x_1}{t_2 - t_1}$
\end{enumerate}

\noindent
\Large
Motion at Constant Velocity\\
\normalsize
\noindent
$x = x_0 + vt$
The following computes a new position of $x$ according to an object's initial position ($x_0$), velocity ($v$) and the given time passed ($t$). The equation is a slope-intercept formula.

\noindent
\Large
Velocity at Constant Acceleration\\
\normalsize
\noindent
$v = v_0 + at$
The following computes a new velocity of $v$ according to an object's initial velocity ($v_0$), acceleration ($a$) and the given time passed ($t$). The equation is a slope-intercept formula.

\indent
The following equations can be combined as a system to calculate constant acceleration with velocity, position and time.
When given a final velocity $v_f$ and an initial velocity $v_0$ one can deduce an average velocity ($v$ or $v_{avg}$)\\
$v_f = v_0 + at$ and $x = x_0 + vt \to x = x_0 + (\frac{v_0 + v_f}{2})t \to x = x_0 + \frac{1}{2}(v_0 + v_0 + at)t \to x = x_0 + v_0t + \frac{1}{2}at^2$ after simplification.

\end{document}
