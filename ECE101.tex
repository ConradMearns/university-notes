\documentclass{article}
\usepackage{tikz}
\usepackage[american]{circuitikz}

\usetikzlibrary{arrows,decorations.markings}

\title{Foundations of Electrical and Computer Engineering}
\author{Conrad A. Mearns}

\begin{document}

\maketitle

\noindent
\Large
Elements\\
\normalsize
\begin{enumerate}
  \item {
  Independent Voltage Source\\
  \begin{circuitikz}
    \draw (0,0) to[voltage source] (2,0)
    (4,0) to[voltage source] (2,0);
  \end{circuitikz}\\
  Provides a fixed voltage independent of what is already attached to the circuit.
  }
  \item {
  Independant Current Source\\
  \begin{circuitikz}
    \draw (0,0) to[current source] (2,0)
    (4,0) to[current source] (2,0);
  \end{circuitikz}\\
  Provides a fixed current independent of what is already attached to the circuit as long as there is a closed path.
  }
  \item {
  Dependant (Controlled) Voltage Source\\
  \begin{circuitikz}
    \draw (0,0) to[controlled voltage source] (2,0)
    (4,0) to[controlled voltage source] (2,0);
  \end{circuitikz}\\
  Provides a voltage dependant on some other voltage or current.
  }
  \item {
  Dependant (Controlled) Current Source\\
  \begin{circuitikz}
    \draw (0,0) to[controlled current source] (2,0)
    (4,0) to[controlled current source] (2,0);
  \end{circuitikz}\\
  Provides a current dependant on some other voltage or current.
  }
  \item {
  Resistor\\
  \begin{circuitikz}
    \draw (0,0) to[resistor] (2,0);
  \end{circuitikz}\\
  Unit: Ohm ($\Omega$)\\
  Absorbs power and follows Ohm's Law. $V = Ri$
  }
\end{enumerate}

\noindent
\Large
Analog Circuit Fundementals\\
\normalsize
\noindent

\begin{enumerate}
  \item {
  Current $\equiv$ the flow of charge around a closed path in a circuit.\\
  Symbol $\equiv$ $i(t)$ or $i$\\
  Unit $\equiv$ Amperes, Amps or (A)\\
  Current is a vector of Amperes and direction. It may be defined at any one point along a circuit with either positive or negative values. A negative value current is equivalent to a positive value but opposite direction current.
  }
  \item {
  Voltage $\equiv$ the measure of potential difference between two points in a circuit.\\
  Symbol $\equiv$ $v(t)$ or $v$\\
  Unit $\equiv$ Volts or (V)\\
  Voltage is a vector of of voltage and polarity. The point with postive polarity is the point where the potential difference is greatest. Voltage may also be either positive or negative, and a negative voltage is equivalent to a positive voltage with reversed polarity.
  }
  \item {
  Power $\equiv$ a measure of useful output of a circuit.\\
  Symbol $\equiv$ $p(t)$ or $p$\\
  Unit $\equiv$ Watts or (W)\\
  Relationship $\equiv$ $p(t) = v(t) * i(t)$ or $p = vi$\\
  Conservation of Power $\equiv$ in a valid circuit, the total power supplied is equivalent to the total power absorbed.
  Power can be absorbed or supplied by a given element. Circuit validity, as used in the of Conservation of Power is true as a consequence of the law. That is to say, a circuit that does not satisfy the Conservation of Power is not a valid circuit.
  }
\end{enumerate}

\noindent
To determine if an element absorbs or supplies power:
\begin{enumerate}
  \item Take measurements of current and voltage before and after the element, in terms of positive values.
  \item {
  If current flows into the positive side $\to$ the element absorbs power.\\
  If current flows into the negative side $\to$ the element supplies power.
  }
\end{enumerate}

\noindent
\Large
Kirchhoff's Laws\\
\normalsize
\noindent
\begin{enumerate}
  \item {
  Node $\equiv$ a node is a point in a circuit where two or more elements are connected. All points along a wire are the same node.
  }
  \item {
  Loop $\equiv$ a loop in a circuit is a closed path that begins and ends at the same node and goes through at least one element.
  }
\end{enumerate}

\noindent
Kirchhoff's Current Law (KCL)\\
\indent
The total current into a node is equal to the total current out of the node.

\noindent
Kirchhoff's Voltage Law (KCL)\\
\indent
The sum of the voltages around a loop in the circuit is zero.\\
When tracing the loop, if voltage goes from negative to positive through an element, the voltage is positive. If voltage goes from positive to negative, the voltage is negative.

\noindent
Both laws must be true in order to have a valid circuit.

\noindent
\Large
Parallel Elements and Elements in Series\\
\normalsize
\noindent
Series $\equiv$ two elements are in series if they share 1 node and no other element is attached to that node. Elements in series always have the same current.\\
Parallel $\equiv$ two elements are parallel if they share 2 nodes. Parallel elements have the same voltage.

\noindent
\Large
Analyzing All-Series / Single Loop Circuits with Resistors\\
\normalsize
\begin{enumerate}
  \item Label all elements with same current
  \item Label Voltage across each element as variables
  \item Calculate KVL for the loop
  \item Calculate Ohm's Law for each resistor
  \item Substitute Ohm's Law equation back into KVL equation
  \item Modify the equation to solve for a single unknown (which is often the current)
  \item Solve for the unknown
  \item Use Ohm's Law to get voltages
  \item Calculate power
\end{enumerate}

\end{document}
