\documentclass{article}
\usepackage{tikz}

\usetikzlibrary{arrows,decorations.markings}

\title{Foundations of Electrical and Computer Engineering}
\author{Conrad A. Mearns}

\begin{document}

\maketitle

\noindent
\Large
Analog Circuit Fundementals\\
\normalsize
\noindent

\begin{enumerate}
  \item {
  Current $\equiv$ the flow of charge around a closed path in a circuit.\\
  Symbol $\equiv$ $i(t)$ or $i$\\
  Unit $\equiv$ Amperes, Amps or (A)\\
  Current is a vector of Amperes and direction. It may be defined at any one point along a circuit with either positive or negative values. A negative value current is equivalent to a positive value but opposite direction current.
  }
  \item {
  Voltage $\equiv$ the measure of potential difference between two points in a circuit.\\
  Symbol $\equiv$ $v(t)$ or $v$\\
  Unit $\equiv$ Volts or (V)\\
  Voltage is a vector of of voltage and polarity. The point with postive polarity is the point where the potential difference is greatest. Voltage may also be either positive or negative, and a negative voltage is equivalent to a positive voltage with reversed polarity.
  }
  \item {
  Power $\equiv$ a measure of useful output of a circuit.\\
  Symbol $\equiv$ $p(t)$ or $p$\\
  Unit $\equiv$ Watts or (W)\\
  Relationship $\equiv$ $p(t) = v(t) * i(t)$ or $p = vi$\\
  Conservation of Power $\equiv$ in a valid circuit, the total power supplied is equivalent to the total power absorbed.
  Power can be absorbed or supplied by a given element. Circuit validity, as used in the of Conservation of Power is true as a consequence of the law. That is to say, a circuit that does not satisfy the Conservation of Power is not a valid circuit.
  }
\end{enumerate}

\noindent
To determine if an element absorbs or supplies power:
\begin{enumerate}
  \item Take measurements of current and voltage before and after the element, in terms of positive values.
  \item {
  If current flows into the positive side $\to$ the element absorbs power.\\
  If current flows into the negative side $\to$ the element supplies power.
  }
\end{enumerate}

\end{document}
